
%%%%%%%%%%%%%%%%%%%%%%%%%%%% Document Setup %%%%%%%%%%%%%%%%%%%%%%%%%%%%
\documentclass[10pt]{article}
% This is a helpful package that puts math inside length specifications
\usepackage{calc}

% Layout: Puts the section titles on left side of page
\reversemarginpar

%% Letter sized paper
\usepackage[paper=letterpaper,
            %includefoot, % Uncomment to put page number above margin
            marginparwidth=1.2in,     % Length of section titles
            marginparsep=.05in,       % Space between titles and text
            margin=1in,               % 1 inch margins
            includemp]{geometry}

%% Use these lines for A4-sized paper
%\usepackage[paper=a4paper,
%            %includefoot, % Uncomment to put page number above margin
%            marginparwidth=30.5mm,    % Length of section titles
%            marginparsep=1.5mm,       % Space between titles and text
%            margin=25mm,              % 25mm margins
%            includemp]{geometry}

%% More layout: Get rid of indenting throughout entire document
\setlength{\parindent}{0in}

%% This gives us fun enumeration environments. compactenum will be nice.
\usepackage{paralist}

%% Reference the last page in the page number
%
% NOTE: comment the +LP line and uncomment the -LP line to have page
%       numbers without the ``of ##'' last page reference)
%
% NOTE: uncomment the \pagestyle{empty} line to get rid of all page
%       numbers (make sure includefoot is commented out above)
%
\usepackage{fancyhdr,lastpage}
\pagestyle{fancy}
%\pagestyle{empty}      % Uncomment this to get rid of page numbers
\fancyhf{}\renewcommand{\headrulewidth}{0pt}
\fancyfootoffset{\marginparsep+\marginparwidth}
\newlength{\footpageshift}
\setlength{\footpageshift}
          {0.5\textwidth+0.5\marginparsep+0.5\marginparwidth-2in}
\lfoot{\hspace{\footpageshift}%
       \parbox{4in}{\, \hfill %
                    \arabic{page} of \protect\pageref*{LastPage} % +LP
%                    \arabic{page}                               % -LP
                    \hfill \,}}

% Finally, give us PDF bookmarks
\usepackage{color,hyperref}
\definecolor{darkblue}{rgb}{0.0,0.0,0.3}
\hypersetup{colorlinks,breaklinks,
            linkcolor=darkblue,urlcolor=darkblue,
            anchorcolor=darkblue,citecolor=darkblue}

%PDF specific settings:
\usepackage[pdftex,
        pdftitle={curriculum vitae},
        pdfauthor={Jason Locklin},
        pdfproducer={pdfLaTeX}]{}

%%%%%%%%%%%%%%%%%%%%%%%% End Document Setup %%%%%%%%%%%%%%%%%%%%%%%%%%%%


%%%%%%%%%%%%%%%%%%%%%%%%%%% Helper Commands %%%%%%%%%%%%%%%%%%%%%%%%%%%%

% The title (name) with a horizontal rule under it
%
% Usage: \makeheading{name}
%
% Place at top of document. It should be the first thing.
\newcommand{\makeheading}[1]%
        {\hspace*{-\marginparsep minus \marginparwidth}%
         \begin{minipage}[t]{\textwidth+\marginparwidth+\marginparsep}%
                {\large \bfseries #1}\\[ -0.15\baselineskip]%
                 \rule{\columnwidth}{1pt}%
         \end{minipage}}


\renewcommand{\section}[2]%
        {\pagebreak[2]\vspace{1.3\baselineskip}%
         \phantomsection\addcontentsline{toc}{section}{#1}%
         \hspace{0in}%
         \marginpar{
         \raggedright \scshape #1}#2}

% An itemize-style list with lots of space between items
\newenvironment{outerlist}[1][\enskip\textbullet]%
        {\begin{enumerate}[#1]}{\end{enumerate}%
         \vspace{-.6\baselineskip}}

% An itemize-style list with little space between items
\newenvironment{innerlist}[1][\enskip\textbullet]%
        {\begin{compactenum}[#1]}{\end{compactenum}}

% To add some paragraph space between lines.
% This also tells LaTeX to preferably break a page on one of these gaps
% if there is a needed pagebreak nearby.
\newcommand{\blankline}{\quad\pagebreak[2]}

%%%%%%%%%%%%%%%%%%%%%%%% End Helper Commands %%%%%%%%%%%%%%%%%%%%%%%%%%%


%%%%%%%%%%%%%%%%%%%%%%%%% Begin CV Document %%%%%%%%%%%%%%%%%%%%%%%%%%%%

\begin{document}
\makeheading{Jason Locklin, B.Sc.}

\section{Contact Information}
%
% NOTE: \rcollength is the width of the right column of the table
%       (adjust it to your liking; default is 1.85in).
%
\newlength{\rcollength}\setlength{\rcollength}{2.5in}%
%
\begin{tabular}[t]{@{}p{\textwidth-\rcollength}p{\rcollength}}
\href{http://psychology.uwaterloo.ca/}%
     {Department of Psychology} & \\
\href{http://www.uwaterloo.ca/}{University of Waterloo}
                           & \textit{Voice:} (519) 888-4567 x36662\\
100 University Ave.            & \textit{Fax:} (614) 292-7596 \\
Waterloo, Ontario           & \textit{E-mail:} \href{mailto:jalockli@uwaterloo.ca}{JaLockli@UWaterloo.ca}\\
N2L 3G1    & \textit{WWW:}
\href{http://artsweb.uwaterloo.ca/~jalockli}{artsweb.uwaterloo.ca/\~{}jalockli}\\
\end{tabular}

\section{Research Interests}  %comment out for teaching
%
Visuomotor Control, Perception, Brain Injury, Object Tracking.

\section{Education}
%
\href{http://www.uwaterloo.ca/}{{University of Waterloo}}

\begin{outerlist}

\item[] \href{http://psychology.uwaterloo.ca}
             {\textbf{M.A., Behavioural and Cognitive Neuroscience}}
        (expected graduation date: June 2009)
        \begin{innerlist}
        \item Thesis Topic: Effects of Concussion on Fine Motor Control
        \item Adviser:
              \href{http://psychology.uwaterloo.ca/people/faculty/jdancker/index.html}
                   {Professor James Danckert}
        \item Area of Study: Fine Motor Control, Concussion Research.
        \end{innerlist}

\item[] \href{http://science.uwaterloo.ca}
        {\textbf{B.Sc. (Honours)}},
             Major: Psychology, Minor: Biology, June 2007
        \begin{innerlist}
        \item Course load including strong mix of Psychology and Natural Science: Biology, Physics, Chemistry, BioChemistry, Organic Chemistry, and Calculus.
        \item Achieved a 95\% in Advanced Data Analysis.
        \item Average GPA over final 4 terms 84\%.
        \end{innerlist}

\end{outerlist}

\section{Awards}
%
\href{http://www.uwaterloo.ca}{University of Waterloo}
\begin{innerlist}
\item Arts Graduate Enhancement Scholarship, 2007--2008
\item Dean of Science Honours List, 2004
\item Dean of Science Honours List, 2005
\end{innerlist}

\blankline
\begin{innerlist}
\item Aiming for the Top Tuition Scholarship, 2003
\end{innerlist}

\section{Teaching Experience}
\href{http://psychology.uwaterloo.ca}{Department of Psychology}, University of Waterloo:

\begin{outerlist}

\item[] \textbf{Basic Data Analysis}: \textit{Teaching Assistant}
        \hfill \textbf{Jan. 2008 to Apr. 2008}
\begin{innerlist}
\item Instruct a weekly tuturial for 30 students, which consisted of a 30 minute review lecture of the week's topic, and 30 minutes of practical instruction on solving data analytic problems.
\item Develop weekly tuturial lesson plans in cooporation with other TAs.
\item Create marking keys for tests and grade students.
\item Course Description:\\ \texttt{An introduction to the logic and methods of descriptive and inferential statistics with emphasis on application in Psychology. Topics covered include measures of central tendency and variability, distributions, the normal distribution, z-scores, hypothesis testing, probability, chi-square tests, t-tests, power, and correlation and regression.}
\end{innerlist}
\blankline

\item[] \textbf{Advanced Data Analysis}: \textit{Teaching Assistant}%
        \hfill \textbf{Sept. 2007 to Dec. 2007}
\begin{innerlist}
\item Develop and lead regular 1 hour tutorials instructing 30 students to utilize the statistical software package SPSS in analyzing real world experimental and observational data.
\item Graded weekly assignments and tests.
\item Course Description: \\ \texttt{Aimed at developing an understanding of the use and interpretation of statistics in complex research designs. Emphasis on analysis of variance and multiple comparison techniques to interpret the results of multi-factor experiments. The importance of power in factorial designs will be discussed. The course includes a computer component that ties the use of a statistical package to the topics discussed in lectures.}
\end{innerlist}
\end{outerlist}


\section{Research Experience}
\href{http://psychology.uwaterloo.ca}{Department of Psychology}, University of Waterloo:

\begin{outerlist}


\item[] \textit{Research Assistant (PT)}%
        \hfill \textbf{May 2007 to Current}
\begin{innerlist}
\item Supervisor: Dr. James Danckert
\item Develop a motor-accuracy task for the measurement of concussion symptoms.
\end{innerlist}


\item[] \textit{Research Assistant (FT)}%
        \hfill \textbf{Jan. 2007 to Apr. 2007}
\begin{innerlist}
\item Supervisor: Dr. Jon Fugelsang
\item Develop web-based decision making experiments and collect data.
\end{innerlist}


\item[] \textit{Laboratory Coordinator (PT)}%
        \hfill \textbf{Sep. 2005 to Aug. 2006}
\begin{innerlist}
\item Supervisor: Dr. Scott McCabe
\item Train research assistants, coordinate lab events, oversee several experiments.
\end{innerlist}


\item[] \textit{Research Assistant (PT)}%
        \hfill \textbf{Jun. 2005 to Dec. 2006}
\begin{innerlist}
\item Supervisor: Dr. James Danckert
\item Develop a computer-based task for participants to pursue moving targets on a touch-screen
computer.
\end{innerlist}

\end{outerlist}


\section{Other Relevant Experience}
\href{http://www.uwaterloo.ca}{University of Waterloo}:
\begin{outerlist}
 \item[] \textit{President: Undergraduate Psychology Society}%
        \hfill \textbf{Sep. 2005 to Apr. 2007}
\begin{innerlist}
\item Organize, train and mentor undergraduate psychology students. (volunteer position)
\end{innerlist}
\end{outerlist}



%\section{Publications}

%\section{Books in Preparation}


\section{Conference Publications}
\begin{outerlist}

\item Law, A.*, McCabe, S., Locklin, J., Tan, C., \& Morris, S. (2006),
Perceptions of social rank as a predictor of anger and depression symptoms. Poster session presented at the Graduate Student Research Conference, Waterloo, Ontario, Canada.


\end{outerlist}



\section{Technical Skills}
\begin{outerlist}
\item[] \textit{Statistical / Data Analytical Software}
\begin{innerlist}
 \item Extensive experience with both \href{http://www.spss.com/}{SPSS}, a statistical analysis package for the social sciences, and \href{http://www.r-project.org/}{R}, the most commonly used programming language for statistical computing and graphics.
\end{innerlist}


\item[] \textit{Programming and Scripting Languages}
\begin{innerlist}
 \item Experience using a variety of programming languages (C, Pascal, and JAVA).
 \item Day-to-day familiarity with several scripting languages including Python, UNIX shell scripting (BASH), and BASIC (\href{http://www.pstnet.com/products/e-prime/}{E-Basic}).
 \item Experience using PHP with HTML and CSS in the development of web-based research experiments.
\end{innerlist}


\item[] \textit{Typesetting and Productivity Software}
\begin{innerlist}
 \item Comfortable writing with \TeX{}, \LaTeX{}, and B\textsc{ib}\TeX{} for technical and scientific documents, as well as common office suites such as Microsoft Office and OpenOffice.
\end{innerlist}

\item[] \textit{Data Acquisition Equipment and Software}
\begin{innerlist}
 \item Experienced using AcqKnowlege software in combination with BIOPAC laboratory equipment for physiological measurement (Electromyography (EMG) and Galvanic Skin Response (GSR))
 \item Experience using \href{http://www.pstnet.com/products/e-prime/}{E-Prime} for data collection in Psychology Research.
\end{innerlist}
\end{outerlist}


\section{Mathematical Expertise}
%

Analysis of Variance

\blankline

Multiple Regression/Correlational Analysis

\blankline

Basic Data Analysis including Hypothesis testing, Data Visualization, and Data Reduction

\end{document}

%%%%%%%%%%%%%%%%%%%%%%%%%% End CV Document %%%%%%%%%%%%%%%%%%%%%%%%%%%%%
