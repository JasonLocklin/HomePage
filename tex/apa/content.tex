The division of labour within in the vertebrate visual system has been apparent since \citeA{hartline1938} noted that some retinal ganglion cells in frogs respond selectively to light entering the cell's respective receptive field, while others responded selectively when the light was removed. Further research uncovered retinal ganglion cells selectively responsive to various characteristics of form and motion, leading to a range of explanations \cite{schiller1986}.

\section*{Parallel Visual Systems}
Following the projections of these ganglion cells, \citeA{Livingstone1988} argued that regions of the parvo- and magnocellular layers of the lateral geniculate nucleus (LGNd) are highly specialized for colour, temporal resolution, contrast, and acuity. Furthermore, neurons in the parvocellular region are colour sensitive and are sensitive to high spatial resolution, but they are also slow to respond to change and have low contrast sensitivity. In contrast, magnocellular layers show the opposite pattern \cite{Livingstone1988}. \citeA{Livingstone1988} further suggested that the parvo and magnocellular pathways continue separately through the striate cortex and into the dorsal and ventral visual pathways discovered earlier by \citeA{Ungerleider1982}. %where are dorsal/ventral streams?


Theories based on these two parallel visual systems centred around the idea that the magno system was specialized for spatial organization (ie. the ``where'' system), and the parvo system concentrated on analyzing the scene (ie. the ``what'' system). Unfortunately, as convenient an explanation as this was, more recent data indicates that the magno- and parvocellular layers of the LGNd do not independently feed the dorsal and ventral visual streams \cite{schiller1990}. Spurred by this problem, and the discovery that the in-flight control of action appears to occur rapidly, outside of awareness, and via different mechanisms from conscious representation \citeA{goodale1986}, put fourth a more refined theory regarding the functional distinction between the dorsal and ventral visual streams \cite{milner1995vba}. Rather than inferring the function of the two systems based on visual input characteristics, they chose to differentiate them based on the two rather distinct output functions of the perceptual system.

<...SNIP...>  %email me if you would really like to see the whole paper



\subsubsection*{Clinical Evidence}
<...SNIP...>  %email me if you would really like to see the whole paper

Further support of degraded on-line control in optic ataxia was discovered by \citeA{rossetti2005}. In a delayed pointing task, the target re-appeared just as the participant was to begin the pointing action. Occasionally the new target was placed in a different position, and while healthy people automatically, and without effort, quickly adjusted to the new location, two patients with ataxia  continued to point to the old target location -indicating that they rely exclusively on the pre-existing motor plan \cite{rossetti2005}.
<...SNIP...>  %email me if you would really like to see the whole paper



\section*{Skilled Movement}
Unlike pointing, or grasping towards simple points on a screen, most movement is not simple and requires ``skilled'' movement. For example, movement planning usually requires an assessment of the object properties, and decisions about the appropriate action before the kinemetic plan is prepared. Using fMRI, \citeA{passingham2001cda} found that even the simplest button-pushing action required dorsal stream activation. However, only when the action required a decision based on identity (colour), did the response involve crosstalk between the two streams. In particular, they found that both the inferiotemporal cortex, and the posterior part of the intraparietal sulcus of the left hemisphere was activated in this task \cite{passingham2001cda}. Even simple movements, based solely on the ventral stream information, appear to require the dorsal stream; however, the ventral stream appears to be recruited only when needed. Considering that there appears to be no direct projections from the inferiotemporal cortex to the premotor cortex \cite{sereno1998}, anatomical evidence seems to support the notion that action requires the dorsal stream, and when necessary, it is informed by the ventral stream capabilities.

To further support the idea of ventral mediated, dorsal movement planning, the same pattern should be observed when the action is planned, but never intended to be executed (ie. imagined). When asked to think about performing an action, or view an action performed by another individual, or even to simply view a tool (which has an implied action), regions of the dorsal stream, including the posterior parietal cortex and intraparietal sulcus, are required \cite<For a review, see>{culham2006hpc}. From this, it can be asserted that the planning of actions necessitates the dorsal stream, regardless of the need for motor control.

<...SNIP...>  %email me if you would really like to see the whole paper



\section*{Ideomotor Apraxia}
<...SNIP...>  %email me if you would really like to see the whole paper



\subsubsection*{Planning and On-line Control in IMA}
While, IMA is often referred to as a motor programming disorder, there is too little in the way of known basic cognitive-motor deficits to conclusively say that IMA is a planning or on-line control disorder with certainty \cite{harrington}. Additionally, the current experimental literature is, for the most part, effectively focused on overt errors which can be detected by the human eye in a clinical setting -a restriction that does not favour the observation of on-line control deficits \cite{harrington}. \citeA{wheaton2007} describes orientation errors in correctly rotating the limb to the appropriate plane for the action, and temporal errors manifesting as jerky movements when smooth actions are optimal. Other deficits involve extra movements, or movement using the wrong joints. While using the wrong muscles, or making temporal order errors are indicative of problems developing a motor plan, jerky, or slow and deliberate movements, and problems orienting the hand to appropriately grasp the object are indicative of on-line control problems.

<...SNIP...>  %email me if you would really like to see the whole paper


\section*{Directions for Future Research}
<...SNIP...>  %email me if you would really like to see the whole paper